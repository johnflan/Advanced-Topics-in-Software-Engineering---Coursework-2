\section{Testing}
Due to the importance of this system to the business, it is imperative that we ensure the correctness of the existing and new functionality. Any existing test documents or code for the system is non-existent or has been lost. Before any modifications can be made to the system we must create a comprehensive test framework. Additionally this test infrastructure will help make future changes to this system more quickly and for a lower cost.

Below we will outline our methodology for acceptance and unit testing of the existing code.

\subsection*{Acceptance Testing}
It is important to test as a complete system i.e. for a given range of inputs confirm all the components interact to provide a correct result. Due to the short time-frame and to ensure correctness of the system, it was decided to involve the customer in specification of acceptance tests for the project and if possible have the customer create their own acceptance test suite. 

There are a number of approaches we may use to execute tests provided by non-technical stakeholders. These vary from parsing paragraphs of natural language to simple scripting interfaces and table based system definitions. For a project like the billing system inputs and outputs of tabular data most closely match both our programme requirements and a data format familiar to the AcmeTelecom staff in the billing department.

There are a number of open-source table based acceptance testing tools available, many of which are commonly used and have active communities. We require that the selected tool offers an interface that is simple to use and requires minimal technical knowledge enabling the customer to write parseable acceptance tests. Below we list some of the relevant testing frameworks.


\begin{itemize}
  \item Framework for Integrated Test (FIT)\\
Tests are composed of test documents, usually in HTML to define configuration, input and output parameters for a system. These documents are executed using FIT harnesses to parse the tabular data and execute it against the the system. Finally comparing expected values to the actual values in order to determine if the system meets the requirements, it output reports mirroring the input definitions. 
  \item FitNesse\\
Based on FIT, FitNesse enhances the concept through the use of a wiki environment for writing tests. This promotes increased collaboration, although we still must write fixtures test components.
  \item Concordion\\
An evolution of the FIT concept, Concordion utilises the JUnit framework for executing the tests. It offers cleaner fixture code and explicitly defined mapping through the inclusion of HTML source attributes. But these specification documents will require developer to define the mappings to fixtures.
\end{itemize}

After evaluation of the available acceptance testing tooling, we concluded that Concordion best fulfilled the requirements of the project. 

Implementation of the test fixtures required that some modifications be made to the code. These changes were necessary due to the BillingSystem being tightly coupled with BillGenerator. It was necessary to create a second constructor in the BillingSystem to pass a reference to a BillGenerator. This enabled the fixture to access the results of the call cost calculations and confirm their correctness. Changes like this were also necessary for the CustomerDatabase and TariffDatabase. 


\subsection*{Unit Testing}

Due to the critical nature of the Billing System it is important to ensure regressions are not introduced during development. Before beginning feature development we implemented unit tests across much of the code, this also was helpful for the developers to become familiar with the code-base. Having this test coverage will enable the team to move more quickly and confidently in a test driven framework with the feature development.

Several unit testing frameworks are popular for Java e.g. JUnit and TestNG. These frameworks are quite similar, TestNG having the ability to execute tests in groups and begin slightly more configurable. Due to the teams experience with JUnit and the fact that Concordion integrates with it, JUnit was selected. In addition to the use of JUnit tests we also required a mechanism to ensure that the object under test is calling the correct methods on its dependency's, and to inject test data from those requests. The team members have had good previous experience using JMock and we employed its use in our testing infrastructure.

\subsubsection*{Unit test implementation}
In order to test some some methods it was necessary to provide mock or dummy objects for classes that we did not wish to include in the test. This mocking out of extra functionality enabled more focused and comprehensive testing.

For example in our testing of the BillingSystem.createCustomerBill(), there was an internal call to BillGenerator. We needed to ensure that BillGenerator was called, and confirm that it was called with the correct parameters but not execute any of its functionality. This type of testing resulted in us creating interfaces for many of the classes we were provided with. For the above example we created the BillGeneratorInterface for the corresponding BillGenerator.

Because the code we were provided contained a large portion of its functionality in a compiled JAR, was impossible to create interfaces for the Customer and Tariff types. In this case we created instances of these objects in the unit test and confirmed with JMock that they were being processed correctly.

During the initial unit testing we found several problems with the program’s logic:

\begin{enumerate}
	\item If there are two customers with the same telephone in the database and one of them makes a call, then they are both charged for the call (according to their price plan).
	\item If someone calls himself then he is charged for the call.
	\item If there is a CallStart event from a customer A to the customer B followed by a CallEnd event from customer A to customer C, then customer A is charged for the time between the two events.
	\item If a customer’s price plan changes, then he is charged based on the new price plan for the previous calls he has made, because all call costs are calculated after the change and not when the call has been made.
	\item If the peak and off peak hours are changed while the system is running (there are calls stored in the call log), then the call costs will be calculated based on the new peak/off peak periods.
\end{enumerate}

Once the team was happy with the coverage of the test for the unmodified code, the unit tests were expanded to include the new cases that would occur because of the new feature. 

